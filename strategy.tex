\begin{frame}[fragile]
	\frametitle{Strategy}
	\textbf{Проблема:} написать класс считывающий файл с парами ключ-значения и выполнять их неупорядоченное хранение.
	Класс должен предоставить методы для выполнения поиска значения по ключу. Чтение и поиск должны быть максимально гибкими 
	и адаптируемыми под разные форматы файлов и алгоритмы поиска. 
	
	\vspace{10pt}
	\bf{Приблизительный интерфейс класса:}
	\begin{minted}{c++}
    template <typename key_t, typename value_t>
    class DataHolder{
    public:
      explicit DataHolder(std::string const& path);
      value_t find(key_t const&) const;
    private:
      std::vector<std::pair<key_t, value_t>> data_;
    };
	\end{minted}
\end{frame}

\begin{frame}
  \frametitle{Strategy}
  \textbf{С чего читать:}
  \begin{itemize}
    \item txt/csv/dat etc.
    \item json
    \item binary files (xls, xlsx, bin)
    \item etc...
  \end{itemize} 
  
  \vspace{10pt}
  
  \textbf{Как искать:}
  \begin{itemize}
    \item поиск за $O(N)$
    \item бинарный/тернарный поиск
    \item etc...
  \end{itemize} 

\end{frame}

\begin{frame}[fragile]
  \frametitle{Strategy}
  \begin{minted}{c++}
enum class FileType{
  txt, json, xls
};

template<typename key_t, typename value_t>
DataHolder<key_t, value_t>::DataHolder(
      const std::string &path, FileType type) {
  if (type == FileType::txt) { /*read txt*/ }
  else if (type == FileType::json) { /*read json*/ }
  else if (type == FileType::xls) {/*go panic. read xls. */}
  /*
  * Еще много других else if () ....
  */
}
  \end{minted}  
\end{frame}

\begin{frame}
  \frametitle{Strategy}
  \textbf{Недостатки подхода}:
  \begin{itemize}
    \item Для добавления нового файла необходимо править 2 независимых участка кода (enum и конструктор)
    \item Класс DataHolder становится огромным
    \item Нарушается принцип единственной ответственности
  \end{itemize}  

  \vspace{20pt}
  \textbf{Возможное решение:} перенос чтения файлов на сторонний класс (стратегию) и инкапсуляция его (при необходимости)
\end{frame}

\begin{frame}[fragile]
  \frametitle{Strategy}
  \begin{minted}{c++}
template <typename key_t, typename value_t>
class ReadStrategy{
  public:
  /*Интерфейс класса*/
};
  \end{minted}  
\end{frame}

\begin{frame}[fragile]
  \frametitle{Strategy}
  \begin{minted}{c++}
template <typename key_t, typename value_t>
class ReadStrategy{
  public:
  std::vector<std::pair<key_t, value_t>> 
          read(std::string const& path) {
    return read_impl(path);
  }

  private:
  virtual std::vector<std::pair<key_t, value_t>> 
          read_impl(std::string const& path) = 0;
};
  \end{minted}  
\end{frame}

\begin{frame}[fragile]
  \frametitle{Strategy}
  \begin{minted}{c++}
    template <typename key_t, typename value_t>
    class ReadStrategy{
      public:
      std::vector<std::pair<key_t, value_t>> 
            read(std::string const& path) {
        return read_impl(path);
      }
      virtual ~ReadStrategy() = default;
      private:
      virtual std::vector<std::pair<key_t, value_t>> 
            read_impl(std::string const& path) = 0;
    };
  \end{minted}  
\end{frame}

\begin{frame}[fragile]
  \frametitle{Strategy}
  \begin{minted}{c++}
template<typename key_t, typename value_t>
class TxtReadStrategy: public ReadStrategy<key_t, value_t>{
  private:
  std::vector<std::pair<key_t, value_t>> 
        read_impl(const std::string &path) override {
    /*Implement txt reading*/
  }
};

template<typename key_t, typename value_t>
DataHolder<key_t, value_t>::DataHolder(
    const std::string &path, 
    ReadStrategy<key_t, value_t> *read_strategy) {
  data_ = read_strategy->read(path);
}

/*In main*/
auto reader = TxtReadStrategy<int, double>();
DataHolder<int, double> holder("data.txt", &reader);
  \end{minted}  
\end{frame}

\begin{frame}[fragile]
  \frametitle{Strategy}
  \begin{minted}{c++}
int main(int argc, char** argv){
  auto reader = TxtReadStrategy<int, double>();
  DataHolder<int, double> txt_holder("data.txt", &reader);
  
  auto js_reader = TxtReadStrategy<int, double>();
  DataHolder<int, double> json_holder("data.txt", &js_reader);
  
  auto xls_reader = TxtReadStrategy<int, double>();
  DataHolder<int, double> xls_holder("data.txt", &xls_reader);
  return 0;
}
  \end{minted} 

\bf{Реализация метода find оставлена для самостоятельного написания}
\end{frame}
\begin{frame}[fragile]
  \frametitle{Strategy. А что в compile time?}
  \begin{minted}[fontsize=\footnotesize]{c++}
template<class key_t, class value_t,
  template<class, class> class read_strategy = TxtReadStrategy,
  template<class, class> class search_strategy = NativeSearchStrategy>
class DataHolder {
public:
  explicit DataHolder(std::string const &path,
        read_strategy<key_t, value_t> reader = {},
        search_strategy<key_t, value_t> searcher = {}) 
          : data_{reader.read(path)}, search_s_{std::move(searcher)} {
  }
  
  value_t find(key_t const &key) const {
    return search_s_.find(data_, key);
  }
  
private:
  std::vector<std::pair<key_t, value_t>> data_;
  search_strategy<key_t, value_t> search_s_;
};
  \end{minted}  
\end{frame}

\begin{frame}[fragile]
  \frametitle{Strategy}
  \begin{minted}{c++}
template<typename key_t, typename value_t>
struct TxtReadStrategy {
  std::vector<std::pair<key_t, value_t>> 
        read(std::string const &path) {
    /**Read txt file**/
  }
};

template<typename K, typename V>
struct NativeSearchStrategy {
  V find(std::vector<std::pair<K, V>> const &data, 
               K const &key) const {
    /**Implement native search**/
  }
};
  \end{minted} 
\end{frame}

\begin{frame}[fragile]
  \frametitle{Strategy}
  \begin{minted}{c++}
template<typename key_t, typename value_t>
struct JsonReadStrategy {
  std::vector<std::pair<key_t, value_t>> 
        read(std::string const &path) {
    /**Read json file**/
  }
};

int main(int argc, char** argv){
  DataHolder<int, double> txt_holder("data.txt");
  DataHolder<int, double, JsonReadStrategy> 
                          json_holder("data.txt");
}
  \end{minted} 
\end{frame}

\begin{frame}
  \frametitle{Strategy}
  \textbf{Когда можно использовать:}
  \begin{itemize}
    \item Разные вариации алгоритма в одном объекте
    \item Множество одинаковых классов со слегка разным поведением
    \item Закрытие алгоритмов от данных
    \item Наличие большого количества if определяющего поведение алгоритма
  \end{itemize}
\end{frame}

